\documentclass[12pt]{article}
\usepackage[utf8]{inputenc}
\usepackage{times}
\usepackage{geometry,indentfirst}
\usepackage{amsmath,amssymb,bm}
\usepackage{enumitem}
\usepackage[hidelinks]{hyperref}
\geometry{left=2cm,right=2cm,top=2cm,bottom=2cm}

\title{\textbf{Summary about Prediction Market}}
\author{Zhihao Ruan, Jiayu Yi, Naihao Deng\\\texttt{\{ruanzh,yijiayu,dnaohao\}@umich.edu}}

\begin{document}
\maketitle
\tableofcontents


\newpage
\section{Exponential Family}

\newpage
\section{Scoring Rules}
Scoring rules is the simplest form of prediction mechanism. For every agent with some information, a scoring rule evaluates how close the information is from the actual outcome, and pays the agent in return for his/her information.

\subsection{Motivations}
Taking an event $\mathbf{X}$ with outcome space $\mathcal{X}$, we want to know something about each agent's belief $p(\mathbf{x})$ on the actual outcome $\mathbf{x}\in \mathcal{X}$, compare it with the actual outcome, and see how accurate the agent's prediction is. However, it is impossible to ask agent for his/her entire $p(\mathbf{x})$ probability distribution. What should we do?

We've already known that sufficient statistic is a very suitable parameter to characterize a probability distribution. Therefore, we can use it to represent agent's belief. We just ask each agent for the sufficient statistic $\mathbf{u}(\mathbf{x})$ as a representation of his/her belief $p(\mathbf{x})$. Then we are able to measure how accurate the agent can predict with a scoring rule based on the actual outcome.

Assume $\bm{\hat{\mu}}=\mathbb{E}_p[\mathbf{u}(\mathbf{x})]$ is the \textit{expected} sufficient statistic from agent's belief $p(\mathbf{x})$. Then, with the actual outcome denoted $\mathbf{x}$, the scoring rule takes the form:
\[S(\bm{\hat{\mu}},\mathbf{x}).\]
We can see that the scoring rule is a measure of how close the agent's belief is from the actual outcome.

\subsection{Incentive Compatibility}
\textbf{Incentive compatibility} is a property of prediction mechanism with which the best strategy for an agent to earn the most profit is to \textit{honestly} report all the information as soon as he/she has it. As a prediction mechanism, a proper scoring rule should leverage \textbf{incentive compatibility} in order to get real information from agents.

Assume that we already set the sufficient statistic to be $\mathbf{u}(\mathbf{x})$. Then for each $p\in \mathcal{P}$ that takes this $\mathbf{u}(\mathbf{x})$ as its sufficient statistic, we can calculate its \textit{expected} sufficient statistic $\bm{\mu}=\mathbb{E}_p[\mathbf{u}(\mathbf{x})]$. A scoring rule is thus called \textbf{proper} if it satisfies the following, for all such $p$, for all $\bm{\hat{\mu}}\neq\bm{\mu}$:
\[\mathbb{E}_p[S(\bm{\mu},\mathbf{x})]\geqslant \mathbb{E}_p[S(\bm{\hat{\mu}},\mathbf{x})].\]
Any scoring rule with this property actually encourages agents to report a probability distribution as close to the actual probability distribution of $\mathbf{X}$ as possible, which is in accordance with the essence of incentive compatibility.

\subsection{Logarithmic Scoring Rule}
Suppose that we have set a form of the sufficient statistic. A classic logarithmic scoring rule takes the form:
\[S(\bm{\mu},\mathbf{x})=\ln p(\mathbf{x};\bm{\mu}),\]
where $\bm{\mu}=\mathbb{E}_p[\mathbf{u}(\mathbf{x})]$ is the expected sufficient statistic over $p(\mathbf{x};\bm{\mu})$.

\newpage
\section{Cost Function based Prediction Market with Bayesian Traders}

\end{document}