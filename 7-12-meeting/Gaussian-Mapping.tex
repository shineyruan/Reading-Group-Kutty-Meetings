\documentclass[12pt]{article}
\usepackage[utf8]{inputenc}
\usepackage{geometry, amsmath, times}
\geometry{left=2cm,right=2cm,top=2cm,bottom=2cm}
\begin{document}
\section*{Calculate $p(x;\theta)$ in terms of Gaussian distribution, with known $\sigma^2$}
\begin{align*}
  P\left(x; \mu, \sigma^2\right) & =\frac{1}{\sqrt{2\pi\sigma^2}}e^{-\frac{\left(x-\mu\right)^2}{2\sigma^2}}                                                        \\
                                 & =\frac{1}{\sqrt{2\pi}}\frac{1}{\sigma}e^{-\frac{\left(x-\mu\right)^2}{2\sigma^2}}                                                \\
                                 & =\frac{1}{\sqrt{2\pi}}e^{\ln{\frac{1}{\sigma}}}e^{-\frac{\left(x-\mu\right)^2}{2\sigma^2}}                                       \\
                                 & =\frac{1}{\sqrt{2\pi}}e^{-\ln\left(\sigma\right)-\frac{\left(x-\mu\right)^2}{2\sigma^2}}                                         \\
                                 & =\frac{1}{\sqrt{2\pi}}e^{-\frac{x^2}{2\sigma^2}+\frac{\mu x}{\sigma^2}-\frac{\mu^2}{2\sigma^2}+\ln\left(\frac{1}{\sigma}\right)} \\
                                 & =\frac{1}{\sqrt{2\pi}}e^{\left[
    \frac{\mu}{\sigma^2}\quad-\frac{1}{2\sigma^2}\right]
  \left[x\quad x^2\right]^T-\left(\frac{\mu^2}{2\sigma^2}+\ln\sigma\right)}
\end{align*}

Therefore
\begin{equation}
  P\left(x; \mu, \sigma^2\right)= \frac{1}{\sqrt{2\pi}}\exp\left( {\begin{bmatrix}
      \frac{\mu}{\sigma^2} \\-\frac{1}{2\sigma^2}
    \end{bmatrix}\cdot\begin{bmatrix}
      x \\x^2
    \end{bmatrix}
    -\left(\frac{\mu^2}{2\sigma^2}+\ln\sigma\right)} \right)
  \label{final_form}
\end{equation}

In Eq (\ref{final_form}), counting measure is $1/\sqrt{2\pi}$, natural parameter $\eta$ is $\begin{bmatrix}
    \frac{\mu}{\sigma^2} \\-\frac{1}{2\sigma^2}
  \end{bmatrix}$, sufficient statistic $t(x)$ is $\begin{bmatrix}
    x \\x^2
  \end{bmatrix}$, log-partition function $a(\eta)$ is $\left(\frac{\mu^2}{2\sigma^2}+\ln\sigma\right)$. Since \[\eta_1 =\frac{\mu}{\sigma^2}\qquad \eta_2 =-\frac{1}{2\sigma^2},\] thus,
$$
  \mu=\frac{\eta_1}{-2\eta_2}.
$$

And,
\begin{align*}
  a(\eta) & =\left[\frac{\left(\frac{\eta_1}{2\eta_2}\right)^2}{2\left(\frac{1}{-2\eta_2}\right)^2}+\ln\sqrt{\frac{1}{-2\eta_2}}\right]=-\frac{1}{2}\eta_1^2\frac{1}{2\eta_2}-\frac{1}{2}\ln\left(-2\eta_2\right) \\
          & =\frac{\eta_1^2}{4\eta_2}-\frac{1}{2}\ln\left(-2\eta_2\right)                                                                                                                                         \\
\end{align*}

\newpage
\section*{Calculate $p(\theta;b_0)$ in terms of Gaussian distribution, with known $\sigma^2$}

In this setting, the prior becomes $P\left( \mu;m,v^2 \right)$ since $\sigma$ is known. We have
\begin{align*}
  P\left( \mu;m,v^2 \right) & =\frac{1}{\sqrt{2\pi}v}\exp\left( -\frac{(\mu-m)^2}{2v^2} \right)                                                                      \\
                            & =\frac{1}{\sqrt{2\pi}}\exp\left( -\ln v-\frac{(\mu-m)^2}{2v^2} \right)                                                                 \\
                            & =\frac{1}{\sqrt{2\pi}}\exp\left( -\ln v-\frac{\mu^2}{2v^2}+\frac{m}{v^2}\mu-\frac{m^2}{2v^2} \right)                                   \\
                            & =\frac{1}{\sqrt{2\pi}}\exp\left( \begin{bmatrix}
      \frac{m}{v^2} \\-\frac{1}{2v^2}
    \end{bmatrix}\cdot\begin{bmatrix}
      \mu \\\mu^2
    \end{bmatrix}-\left( \ln v+\frac{m^2}{2v^2} \right) \right)
\end{align*}

Therefore we have \[P\left( \mu;m,v^2 \right)=\frac{1}{\sqrt{2\pi}}\exp\left( \begin{bmatrix}
      \frac{m}{v^2} \\-\frac{1}{2v^2}
    \end{bmatrix}\cdot\begin{bmatrix}
      \mu \\\mu^2
    \end{bmatrix}-\left( \ln v+\frac{m^2}{2v^2} \right) \right).\]
Since $P(\mu;m,v^2)$ can also be parametrized by
\begin{align*}
  P(\mu;n,\nu) & =\exp\left( \langle n\nu,\theta\rangle+nT(\theta)-\psi(\nu,n) \right)                      \\
               & =\exp\left( \begin{bmatrix}
    n\nu \\n
  \end{bmatrix}\cdot\begin{bmatrix}
    \theta \\T(\theta)
  \end{bmatrix}-\psi(\nu,n) \right),
\end{align*} we can see that
\begin{align*}
  n           & =-\frac{1}{2v^2},         \\
  \nu         & =-2m                     \\
  \psi(\nu,n) & =\ln v+\frac{m^2}{2v^2}.
\end{align*}


\end{document}
