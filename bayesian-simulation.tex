\documentclass[12pt]{article}
\usepackage[utf8]{inputenc}
\usepackage{times}
\usepackage{geometry,indentfirst}
\usepackage{amsmath}
\usepackage{enumitem}
\geometry{left=2cm,right=2cm,top=2cm,bottom=2cm}

\begin{document}
\begin{center}
    \textbf{\large Procedures for Bayesian Trader Prediction Market Simulation}
\end{center}

\begin{itemize}
    \item Hyperparameters:
          \begin{itemize}[label=$\circ$]
              \item A random event $X$ with all kinds of outcome $x$ and some probability distribution $p$;
              \item Agent's belief for this event, $p(x;\theta)$. This belief has a general form: \[ p(x;\theta)=\exp\left( \langle\theta,\phi(x)\rangle-T(\theta) \right), \] $T(\theta)$ as the log-partition function.
                    \\\textbf{\textit{Note.} Agent's belief on the probability distribution of this event is not reflected directly from $p(x;\theta)$. Instead, it will be reflected from its prior, i.e, the distribution of the parameter $\theta$.}
              \item Agent's prior for this event. Prior should be of the form \[ p(\theta;b_0)=\exp\left( \langle n\nu,\theta\rangle+nT(\theta)-\psi(\nu,n) \right), \] where $b_0=\begin{bmatrix}
                            n\nu \\n
                        \end{bmatrix}$.
          \end{itemize}
    \item Procedures:
          \begin{enumerate}
              \item According to the current outstanding shares and the cost function, assume the current market price for the security (contract) is $$\nabla C(\theta)=\nu.$$
              \item One agent comes into the market with some prior $p(\theta;b_0)$ where $b_0=\begin{bmatrix}
                            n\nu \\n
                        \end{bmatrix}$.
              \item He is provided with a \textit{private} set of data points of size $m$ and mean $\hat{\mu}$.
              \item His posterior belief $p(\theta;b_1)$ is updated to be of same form but $b_1=\begin{bmatrix}
                            n\nu+m\hat{\mu} \\n+m
                        \end{bmatrix}$.
              \item He would like to buy/sell some number $\delta$ of security (contract) such that $$\nabla C(\theta+\delta)=\dfrac{n\nu+\hat{\mu}}{n+1}.$$
              \item Repeat the above steps until all agents have traded in the market.
          \end{enumerate}
    \item Questions and concerns:
          \begin{enumerate}
              \item Why is the fact that every agent would have a prior with a parameter such that \textbf{its first entry is always $\nu$ times larger than the second entry?}
          \end{enumerate}
\end{itemize}

\end{document}