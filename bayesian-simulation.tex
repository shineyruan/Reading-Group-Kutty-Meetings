\documentclass[12pt]{article}
\usepackage[utf8]{inputenc}
\usepackage{times}
\usepackage{geometry,indentfirst}
\usepackage{amsmath}
\usepackage{enumitem}
\geometry{left=2cm,right=2cm,top=2cm,bottom=2cm}

\begin{document}
\begin{center}
    \textbf{\large Procedures for Bayesian Trader Prediction Market Simulation}
\end{center}

\begin{itemize}
    \item Hyperparameters:
          \begin{itemize}[label=$\circ$]
              \item A random event $X$ with all kinds of outcome $x$ and some probability distribution $p$;
              \item Market maker's belief for this event, $p(x;\theta)$. This belief has a general form: \[ p(x;\theta)=\exp\left( \langle\theta,\phi(x)\rangle-T(\theta) \right), \] $T(\theta)$ as the log-partition function.
              \item Market maker's prior for this event. Prior should be of the form \[ p(\theta;b_0)=\exp\left( \langle n\nu,\theta\rangle+nT(\theta)-\psi(\nu,n) \right), \] where $b_0=\begin{bmatrix}
                  n\nu\\n
              \end{bmatrix}$.
              \item Posterior update: \[ \mathbf{E}_{\theta\sim b_0}\mathbf{E}_{x\sim\theta}[\phi(x)]=\nu, \] therefore $\nu$ is the posterior mean, because posterior is essentially $p(\theta|x)$. \textit{With regards to cost-function prediction market, we can also see that $\nu$ is the price.}
          \end{itemize}
    \item Procedures:
          \begin{enumerate}
              \item One agent comes into the market and purchase some amount of contracts (securities). Assume the agent possess dataset of size $m$ and mean $\hat{\mu}$.
              \item Update the posterior mean (the new price): \[\nu'=\frac{n\nu+m\hat{\mu}}{n+m}.\]
              \item Repeat the above steps until all agents have traded in the market.
          \end{enumerate}
\end{itemize}

\end{document}